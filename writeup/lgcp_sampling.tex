\documentclass[review]{elsarticle}

\usepackage{lineno,hyperref,amsmath,amsfonts,subcaption}
\modulolinenumbers[5]

\journal{Spatial Statistics}

%%%%%%%%%%%%%%%%%%%%%%%
%% Elsevier bibliography styles
%%%%%%%%%%%%%%%%%%%%%%%
%% To change the style, put a % in front of the second line of the current style and
%% remove the % from the second line of the style you would like to use.
%%%%%%%%%%%%%%%%%%%%%%%

%% Numbered
%\bibliographystyle{model1-num-names}

%% Numbered without titles
%\bibliographystyle{model1a-num-names}

%% Harvard
\bibliographystyle{model2-names.bst}\biboptions{authoryear}

%% Vancouver numbered
%\usepackage{numcompress}\bibliographystyle{model3-num-names}

%% Vancouver name/year
%\usepackage{numcompress}\bibliographystyle{model4-names}\biboptions{authoryear}

%% APA style
%\bibliographystyle{model5-names}\biboptions{authoryear}

%% AMA style
%\usepackage{numcompress}\bibliographystyle{model6-num-names}

%% `Elsevier LaTeX' style
%\bibliographystyle{elsarticle-num}
%%%%%%%%%%%%%%%%%%%%%%%

\begin{document}

\begin{frontmatter}

\title{Log-Gaussian Cox processes and sampling paths: towards optimal design}

%% Group authors per affiliation:
\author[msuaddr]{Kenneth Flagg\corref{mycorrespondingauthor}}
\cortext[mycorrespondingauthor]{Corresponding author}
\ead{kenneth.flagg@montana.edu}

\author[msuaddr]{John Borkowski}
\author[msuaddr]{Andrew Hoegh}

\address[msuaddr]{Department of Mathematical Sciences, Montana State University, Bozeman, MT 59717}

\begin{abstract}

\paragraph{Goal of this paper (placeholder abstract)} Evaluate a wide variety
of path designs in terms design-based heuristics and model-based criteria for
spatial prediction using Bayesian LGCP models. Identify promising initial
designs for later optimization and sequential design (not actually optimizing
yet). Illuminate any relationships among design characteristica and
predictive criteria that will be helpful for constrained optimization. Discuss
sequential contruction of paths as a precursor to online sequential design.

\end{abstract}

\begin{keyword}
log-Gaussian Cox process\sep optimal sampling\sep model-based design\sep spatial sampling design
%\MSC[2010] 00-01\sep  99-00
\end{keyword}

\end{frontmatter}

\linenumbers

% Can name paragraphs with \paragraph{Title}


\section{Introduction}
% State the objectives of the work and provide an adequate background, avoiding a detailed literature survey or a summary of the results.

Spatial point process models have long been considered generally infeasible
because of their computational demands, but recent advances in Bayesian
computing have made the Log-Gaussian Cox process an attainable model in
practice~\citep{rueetal, lindgrenetal, illianetal, simpsonetal}. Variable
sampling effort leads to a degraded point pattern~\cite{chakrabortyetal} and
it is relatively simple to accomodate variable sampling effort in these models
using modern computing tools~\citep{yuanetal}. However, the literature on
optimal sampling for spatial point process models is in its
infancy~\citep{liuvanhatalo}. In this article, we present a variety of sampling
path designs and assess their optimality for LGCP models.

Point pattern data are routinely collected in species distribution studies and
ordnance response projects. These applications may use quadrat sampling or
line-transect sampling, with transect sampling being more common. When the
objective is mapping where events occur in space, various spatial mapping
procedures have been used. Traditionally these have involved aggregating the
data to grid cell counts or computing moving averages. Aggregation has the
downside of introducing arbitrary structure into the data by the choice of
gridding scheme or averaging window, and requires uneccessary computation
effort~\citep{simpsonetal}. Software is now available to fit spatial point
process models to data acquired via distance sampling and simultaneously
estimate the detection function~\citep{dspat,baser}.
% Recent example of aggregation: https://link-springer-com.proxybz.lib.montana.edu/content/pdf/10.1007/s13253-020-00386-3.pdf

In ecological settings, sampling plans are often designed around the goal of
estimating total abundance. Ordnance response surveys are typically designed
with the objective of detecting (but not necessarily mapping) intensity
hotspots~\citep{em200-1-15,flaggetal}. However, to our knowledge, there has
been very little work done in deciding where to collect data when the goal is
to map the intensity using a spatial point process model. In this paper, we
adapt nearest-neighbor criteria to the path design setting and introduce a
sequential path construction scheme as a starting point for future
optimization. We then compare a variety of path design schemes with respect
to a suite of model-based and design-based criteria for simulated point
pattern data.


\subsection{Spatial design}

\paragraph{Design-based sampling}
Most classical sampling work has been done for points or small quadrats
approximated as points. Space-filling criteria may be good starting
points~\citep{borkowskipiepel}. Latin hypercube sampling has space-filling
properties~\cite{mckayetal,husslageetal}.

\paragraph{Space-filling curves}
Used in design of dense or stretchable circuits~\citep{ogorzalek,mazhang} and
high-dimensional data visualization in bioinformatics~\citep{hilbertvis}. Peano
curve is very flexible for filling irregular shapes~\citep{fanetal}.

Space-filling curves are one-dimensional paths constructed iteratively; as the
number of iterations goes to infinity, the limiting path has nonzero area and
actually fills the space~\citep{sagan}. For applications we stop after a finite
number of iterations. The Hilbert curve is fast and simple to construct.

\paragraph{Model-based spatial design}
Regularity is optimal for spatial prediction but randomness and a variety of
interpoint distances are best for parameter estimation~\citep{diggle}.
Inhibitory plus close pairs is a good compromise~\citep{chipetaetal2017}.


\subsection{Paths as sampling designs}

While some ideas about the characteristics of a good point design apply to
paths, creating an optimal path design is not as simple as connecting the
points of a point design with line segments. There are many ways to connect
points into a path, so optimal design criteria must apply to the whole path and
not only to the waypoints.

\citet{pollard} adaptively zigzagged their line transects in a species
abundance survey.

The Visual Sample Plan software includes features to create systamatic transect
plans and augment plans with additional transects in regions lacking spatial
coverage~\citep{vspguide}. It helps the user choose the transect spacing to
maximize the probability of detecting the presence of a hotspot of specified
size and intensity. However, it does not employ criteria to optimize spatial
prediction.

\citet{liuvanhatalo} used narrow quadrats (swaths along line-transects) as
their sampling units. The transects were short relative to the size of the
study region and not connected into a path.


%\subsection{Multi-objective optimization}

%summarize \citet{lark} and related

%we can use a large suite of criteria to explore the relationships among them

% (Don't need for this paper.)


\section{Materials and methods}
% Provide sufficient details to allow the work to be reproduced by an independent researcher. Methods that are already published should be summarized, and indicated by a reference. If quoting directly from a previously published method, use quotation marks and also cite the source. Any modifications to existing methods should also be described.

Heuristics of a good path design:
\begin{itemize}
\item Should start with a sparse design with regular spacing, then refine with
infill
\begin{itemize}
\item Provides good spatial coverage even if aborted early
\item Imagine downloading a high-resolution intensity jpeg over 56k
\end{itemize}
\item Path should avoid sharp turns but is allowed to cross itself
\item One option is to generate two segments at a time, first a short-to-medium
length segment to get to the start of the next transect, then a  medium-to-long
segment for the transect
\item Could have new segment length be negatively correlated with the previous
segment length
\end{itemize}


\subsection{Design-based criteria}

We give attention to some design-based criteria that tie directly to practical
considerations of data collection.

\paragraph{Path length}
The total distance that traveled is often a constraint. Minimize it.

\paragraph{Corners}
Data collection equipment (e.g. metal detectors) may have limited mobility,
requiring minimizing the nubmer or angle of turns.

\paragraph{Nearest neighbor distance}
A common criterion for space-filling designs, we adapt it to be meaningfully
calculated for any point on a path. Define the \(k\)th-order nearest neighbor
distance as \(\mathrm{nnd}_{k}(u) = \min|u - v|\) where \(v\) is any point
in the set of path segments at least \(k\) steps away from the segment
containing \(u\). If \(k = 0\), this includes \(v\) in the same segment as
\(u\) so trivially \(\mathrm{nnd}_{0}(u) = 0\) for all \(u\) in the path.
\(\mathrm{nnd}_{1}(u)\) includes all segments except the one containing \(u\).
\(\mathrm{nnd}_{2}(u)\) excludes the segment containing \(u\) and segments with
which it shares vertices. Segments not accessible by a connected path starting
at \(u\) are always included. Maximize \(\min[\mathrm{nnd}_{2}(u)]\),
\(\mathrm{avg}[\mathrm{nnd}_{2}(u)]\), and
\(\mathrm{avg}[\mathrm{nnd}_{1}(u)]\).

{\it (Move details to appendix.)}


\subsection{Model-based criteria}

\paragraph{Mean-squared prediction error}
Minimize MSPE for the GP.

\paragraph{Posterior prediction variance}
Minimize maximum prediction variance and average prediction variance for GP.

%\paragraph{Posterior parameter variance}
%Minimize posterior variance for each model parameter (intercept, variance,
%spatial scale).

%\paragraph{Decision-based criteria}
%Maximize error rates and AUC of thresholding the intensity at an action level.
%(Leaving for future work.)


\subsection{Sampling schemes}

{\it These are the focus, move them earlier?}

\paragraph{Parallel line transects} Parallel straight-line transects are common
in ordnance response studies and in ecological studies using distance sampling.
Systematic designs are common because they pprovides good spatial coverage in
the sense that any point in the study region has a known maximum distance from
the path. For point designs, systematic designs are optimal for prediction,
simple random samples are optimal for estimation, and inhibitory with close
pair designs are becomming a popular compromise. We adapt all of these to the
parallel line transect setting. We use line transects running north-south, with
three ways of choosing the horizontal coordinate: simple random sample (SRS),
systematic with a random starting point and even spacing, inhibitory plus close
pairs. To vary the length, we use designs with 10, 25, 50, and 70 transects.
Figure~\ref{linexsects} shows an example of each scheme with 25 transects.

{\it (note about inhib plus pairs)}

{\it (note about isotropy)}

\begin{figure}
\includegraphics[width=5in]{SRS000176.pdf}

\includegraphics[width=5in]{Sys000141.pdf}

\includegraphics[width=5in]{Inhib000171.pdf}

\caption{Examples of three different parallel line transect designs with the
same number of transects.}
\label{linexsects}
\end{figure}

\begin{figure}
\includegraphics[width=5in]{Serp000124.pdf}

\includegraphics[width=5in]{Serp000539.pdf}

\caption{Examples of systematic serpentine transect designs of the same
distance. The number of zigzags is the number of north-south segmetns per
transect.}
\label{serps}
\end{figure}

\paragraph{Latin hypercube sampling}
Random Latin hypercube design connected by shortest path. Waypoints generated
by the \texttt{lhs} R package~\cite{lhs}. Connected into a the shortest path
by the \texttt{TSP} package~\citep{tsp}.

\begin{figure}
\includegraphics[width=5in]{LHS-TSP000161.pdf}
\caption{Example of a shortest path through a Latin hypercube smapling design.}
\label{lhstsp000259}
\end{figure}

\paragraph{Space-filling curves}
Hilbert curve generated by HilbertVis package~\citep{hilbertvis}. This is a
deterministic design, so a random offset is added.

\begin{figure}
\includegraphics[width=5in]{Hilbert000180.pdf}
\caption{Example of a Hilbert curve design.}
\label{hilbert000180}
\end{figure}

\paragraph{Particle movement model}
Models the way data are actually collected. Waypoints generated sequentially by
generating a jump distance and a direction. The jump distance is generated from
a scaled beta distribution, and negatively correlated with previous jump
distance. This behavior should approximately alternate between a short
``transition'' and a long transect. The negative correlation was achieved by
applying a \(1 - x\) transformation to a beta autoregressive
process~\citep{mckenzie}. The direction angle is drawn from a bimodal
distribution that is symmetric around 0 (a normal distribution reflectd about
0). {\it explain the Strauss part}

{\it set up to think about adaptive sampling (adding a transect at a time or
stopping early but don't actually do it here)}

\begin{figure}
\includegraphics[width=5in]{RPM001107.pdf}
\caption{Example of a random particle movement design.}
\label{rpm001107}
\end{figure}


\subsection{Model fitting}

INLA~\citep{rueetal}, SPDE~\citep{lindgrenetal}, off-grid~\citep{simpsonetal}


\subsection{Simulation procedure}

We consider a fictitious site \(\mathcal{R}\) with the simple shape of a 1500
unit by 700 unit rectangle. In this site, we will simulate two data generating
models meant to produce random intensity functions with with hotspots. First,
a LGCP with latent GP mean \(\mu = \log(250 / |\mathcal{R}|)\) and a
Mat\'{e}rn covariance with \(\nu = 1\), \(\sigma = 2\), and
\(\text{range} = 200\). This model produces relatively unstructured hotspots
due to large variability in the GP.

Second, the superposition of a two-stage cluster process superposed and a
LGCP. The cluster process (a Neyman-Scott or, more specifically, a Thomas
process) is constructed as follows. The number of clusters is Poisson with
mean 3. The number of events per cluster is Poisson with mean 200. The cluster
centers distributed uniformly over \(\mathcal{R}\). Events come from a
bivariate normal distribution with mean equal to the cluster center and
variance \(\boldsymbol{\Sigma} = \tau^{2}\mathbf{I}\), \(\tau = 50\). The LGCP
has \(\mu = \log(250 / |\mathcal{R}|)\) and Mat\'{e}rn covariance with
\(\nu = 1\), \(\sigma = 1\), and \(\text{range} = 200\). This model is based
upon the typical conceptual model of a firing range, with a background process
(represented by the LGCP) and a small number of higher-intensity foreground
clusters containing the events of interest.

Path design schemes:
\begin{itemize}
\item Simple random sample of north-south line transects
\begin{itemize}
\item \(\text{Number of transects} = 10, 25, 50, 70\)
\item Expect high variance, large prediction error in big gaps.
\end{itemize}
\item Systematic sample of north-south line transects
\begin{itemize}
\item \(\text{Number of transects} = 10, 25, 50, 70\)
\item Uniformly distributed starting point
\item Constant spacing
\item Expect low bias and ok variance, can miss structures at certain sizes,
may not have best space-filling properties.
\end{itemize}
\item Systematic sample of north-south serpentine transects
\begin{itemize}
\item \(\text{Number of transects} = 7, 22, 47, 67\)
\item Uniformly distributed starting point
\item Constant spacing
\item \(\text{Number of zigzags} = 5, 8\)
\item Horizontal zigzag distance set so that the total horizontal distance
traveled equals 2100 units (the length of 3 non-zigzag line transects)
\item Expect better space-filling properties than line-transect designs,
lower bias/variance farther from path, would be better at estimating
anisotropic covariance than line-transects.
\end{itemize}
\item Inhibitory plus close pairs sample of north-south line transects
\begin{itemize}
\item \(\text{Total number of transects (including pairs)} = 10, 25, 50, 70\)
\item \(\text{Number of pairs} = 0.1, 0.2\) times the total number of transects
(rounded up or down to nearest whole number)
\item Pairs uniformly distributed within radius of primaries,
\(\text{max pair radius} = 1500 / \text{total number of transects}\)
\item Position of primaries generated from a 1-dimensional Strauss process with
\(\gamma = 0.05\)
\item A compromise between SRS and systematic in every way.
\end{itemize}
\item Latin Hypercube Sampling waypoints
\begin{itemize}
\item \(\text{Number of bins} = 50, 300, 1200, 2400\)
\item Expect low bias/variance per unit distance traveled, many sharp corners,
some big open areas.
\end{itemize}
\item Hilbert curve
\begin{itemize}
\item \(\text{Order} = 3, 4, 5, 6\)
\item Created in square and then scaled to fit in \(\mathcal{R}\)
\item A uniform random offset added equal to spacing between segments
\item Expect good space filling, good bias and variance, lots of short
segments.
\end{itemize}
\item Random particle movement
\begin{itemize}
\item \(\text{Distance cutoff} = 6700, 17200, 34700, 49700\)
\item Segment lengths uniform 50 to 500 units
\item Adjacent segments uncorrelated or \(\rho = -0.8\)
\item Turn angle \(\mathrm{N}(\mu = \pi / 3, \sigma = \pi / 6)\) or
\(\mathrm{N}(\mu = \pi / 2, \sigma = \pi / 12)\)
\item Angle multiplied by discrete uniform over \(\{-1, 1\}\)
\item Strauss-esque thinning, \(\text{antirepulsion} = 0.8\),
\(\text{pair radius} = 80\)
\item All combinations of the above, plus pair distance of 300 for 6700
distance cutoff
\item Expect variation in all characteristics due to extreme randomness, but
some near-optimality that could be harnessed by search algorithms, should see
exploration followed by infill, negative \(\rho\) with turns centered tightly
on \(\pi / 2\) should mimic zigzagging among parallel transects.
\end{itemize}
\end{itemize}

{\it (Probably should move explanations of schemes to appendix.)}

100 designs from each scheme. All events within a 2 unit radius of the path are
observed. Whole experiment repeated for 5 realizations from each data
generating model.

Model:

\begin{itemize}
\item \(\mathbf{X}\) is a Poisson process on \(\mathcal{R}\) with intensity
\(\lambda(u)\)
\item \(\log[\lambda(u)] = \mu + \mathbf{e}(u)\)
\item \(\mu \sim \mathrm{Unif}(-\infty, \infty)\)
% Or \(\mathrm{N}(0, \infty)\)?
\item \(\mathbf{e}\) is a Gaussian process with mean \(\mathbf{0}\) and a
Mat\'{e}rn covariance function with fixed \(\nu = 1\)
% Remember alpha = nu + d/2 so alpha = 2 and d = 2 imply nu = 1.
\item PC prior on \(\sigma\) and \(\rho\) with \(\mathrm{Pr}(\sigma > 3) = 0.1\)
and \(\mathrm{Pr}(\rho < 100) = 0.1\) \citep{fuglstadetal,simpsonpc}
\item SPDE approach of \citet{lindgrenetal} using mesh in Figure~\ref{meshfull}
\item Likelihood factorization of \citet{simpsonetal}
\end{itemize}

\begin{figure}
\includegraphics[width=5in]{mesh_full.pdf}
\caption{Illustration of the mesh and associated numerical integration scheme
used to approximate the latent GP.}
\label{meshfull}
\end{figure}


%\section{Theory/calculation}
% A Theory section should extend, not repeat, the background to the article already dealt with in the Introduction and lay the foundation for further work. In contrast, a Calculation section represents a practical development from a theoretical basis.


\section{Results}
% Results should be clear and concise.

look at examples of designs that minimize each criterion

look at examples of designs along the Pareto front


\section{Discussion}
% This should explore the significance of the results of the work, not repeat them. A combined Results and Discussion section is often appropriate. Avoid extensive citations and discussion of published literature.

discuss starting points for optimization and sequential design


\section{Conclusions}
% The main conclusions of the study may be presented in a short Conclusions section, which may stand alone or form a subsection of a Discussion or Results and Discussion section.


\appendix
\section{Notation and Terminology}

\begin{itemize}

\item process defined on \(\mathcal{D} \subset \mathbb{R}^{d}\), domain of the
intensity function, in this manuscript \(d = 2\)

\item observation window \(\mathcal{S} \subset \mathcal{D}\)

\item define three regions:
\begin{itemize}
\item the domain \(\mathcal{D}\) over which the process mathematically operates
\item the study region \(\mathcal{R}\) over which inferences are desired
\item the observed/sampled observation window \(\mathcal{S}\)
\end{itemize}

\item general relationship is \(\mathcal{S} \subset \mathcal{R}
\subset \mathcal{D} \subset \mathbb{R}^{d}\) where all of the subset symbols
taken to mean ``subset or equal''

\item \(\mathcal{D}\) can be bounded or unbounded (often equal to
\(\mathbb{R}^{d}\)), $\mathcal{S}$ practically always bounded, \(\mathcal{R}\)
bounded or unbounded depending on application and inferential goals

\item the ``fully surveyed'' (censused) situation is
\(\mathcal{S} = \mathcal{R}\)

\item survey path \(\mathcal{P}\) is a one-dimensional subset of
\(\mathcal{R}\)
\begin{itemize}
\item set of one or more sequences of waypoints connected by line segments
\item \(\mathcal{S}\) is the set of all points within a fixed (and assumed
known) radius of \(\mathcal{P}\)
\end{itemize}

\item \(\mathbf{X}\) point process on \(\mathcal{R}\), \(\mathbf{x} = \{x_{1},
\dots, x_{n}\}\) realized point pattern
\begin{itemize}
\item \(\mathbf{X}_{\mathcal{S}} = \mathbf{X} \cap \mathcal{S}\) the
 restriction of \(\mathbf{X}\) to \(\mathcal{S}\), \(\mathbf{x} = \mathbf{X}
\cap \mathcal{S}\) the realized observeable point pattern
\end{itemize}

\item point \(x \in \mathbf{x}\) called an event

\item intensity function \(\lambda(u)\)

\item types of ``points'' in space:
\begin{itemize}
\item \(x\) event in the point pattern
\item \(s\) numerical integration node
\item \(u\) arbitrary location in \(\mathcal{D}\) used to index intensity
function and predictors
\end{itemize}

\item \(z(u)\) a column vector of covariates/predictors at \(u\) (not used in
this manuscript)

\item ``point'' refers to a \(u\) unless clearly stated otherwise

\item bold for sets and spatial processes, normal italics for spatial vectors

\item \(y\) and variations will be used for objects derived from the point
pattern, e.g. marks, pseudodata

\item distance sampling fits into the framework with expansion of notation
to include a (nontrivial) detection function and differentiate between the
observed and observable point patterns

\end{itemize}


\section{Extension of Nearest Neighbor Distance to Paths}


\section*{References}

\bibliography{lgcp_sampling.bib}

\end{document}
