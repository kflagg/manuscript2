\documentclass[12pt]{article}
\usepackage{amsmath,amsfonts,graphicx,natbib}
\usepackage{subcaption}
\usepackage{lineno}
\usepackage{enumerate}
\usepackage[margin=1in]{geometry}

\usepackage{xcolor}
\usepackage{url}
\def\spacingset#1{\renewcommand{\baselinestretch}
{#1}\small\normalsize} \spacingset{2}

\begin{document}


\title{\bf Log-Gaussian Cox Processes and Line Transect Sampling: Optimal Sampling Critera}

\author{}
\date{}        
\maketitle

\bigskip
\begin{abstract}

\noindent insert abstract here

\end{abstract}

{\it Keywords:} log-Gaussian Cox process, optimal sampling, model-based design, spatial sampling design

\linenumbers


\section{Introduction}

Spatial point processes models have long been considered generally infeasible
because of their computational demands, but recent advances in Bayesian
computing have made the Log-Gaussian Cox process an attainable model in
practice \cite{rueetal, lindgrenetal, illianetal, simpsonetal}. Variable
sampling effort leads to a degraded point pattern \cite{chakrabortyetal} and
it is relatively simple to accomodate variable sampling effort in these models
using modern computing tools \cite{yuanetal}. However, the literature on
optimal sampling for spatial point process models is in its infancy
\cite{liuvanhatalo}.

Point pattern data are routinely collected in species distribution studies and
ordnance response projects. These applications may use quadrat sampling or
line-transect sampling, with transect sampling being more common. When the
objective is mapping where events occur in space, various spatial mapping
procedures have been used. Traditionally these have involved aggregating the
data to grid cell counts or computing moving averages. These have the downside
of introducing arbitrary structure into the data by the choice of gridding
scheme or averaging window, and require uneccessary computation effort
\cite{simpsonetal}. Software is now available to fit spatial point process
models to data acquired via diastance sampling and simultaneously estimate the
detection function \cite{dspat}.

In ecological settings, sampling plans are often designed around the goal of
estimating total abundance. Ordnance response surveys are typically designed
with the objective of detecting (but not necessarily mapping) hotspots.
However, to our knowledge, there has been very little work done in deciding
\emph{where} to collect data when the goal is to map the intensity using a
spatial point process model.


\subsection{Notation and Terminology}

\begin{itemize}

\item process defined on \(\mathcal{D} \subset \mathbb{R}^{2}\), domain of the
intensity function, define \(d = \mathrm{dim}(\mathcal{D})\)

\item observation window \(\mathcal{S} \subset \mathcal{D}\)

\item define three regions:
\begin{itemize}
\item the domain \(\mathcal{D}\) over which the process mathematically operates
\item the study region \(\mathcal{R}\) over which inferences are desired
\item the observed/sampled observation window \(\mathcal{S}\)
\end{itemize}

\item general relationship is \(\mathcal{S} \subset \mathcal{R}
\subset \mathcal{D} \subset \mathbb{R}^{d}\) where all of the subset symbols
taken to mean ``subset or equal''

\item \(\mathcal{D}\) can be unbounded or bounded (often \(\mathbb{R}^{d}\)),
$\mathcal{S}$ practically always bounded, \(\mathcal{R}\) bounded or unbounded
depending on application and inferential goals

\item the ``fully surveyed'' situation is \(\mathcal{S} = \mathcal{R}\)

\item \(\mathbf{X}\) point process on \(\mathcal{R}\), \(\mathbf{x} = \{x_{1},
\dots, x_{n}\}\) realized point pattern

\item point \(x \in \mathbf{x}\) called an event

\item intensity function \(\lambda(u)\)

\item types of ``points'' in space:
\begin{itemize}
\item \(x\) event in the point pattern
\item \(s\) numerical integration node
\item \(u\) arbitrary location in \(\mathcal{D}\) used to index intensity
function and predictors
\end{itemize}

\item \(z(u)\) a column vector of covariates/predictors at \(u\)

\item ``point'' refers to a \(u\) unless clearly stated otherwise

\item bold for sets and spatial processes, normal italics for spatial vectors

\item \(y\) and variations will be used for objects derived from the point pattern,
e.g. marks, pseudodata

\end{itemize}


\section{Model-Based Criteria}

UXO: mapping a site for delineating high-intensity regions
\begin{itemize}
\item minimize time/distance
\item minimize one of these:
\begin{itemize}
\item maximum variance in intensity surface
\item maximum variance in intensity surface \emph{at contours near action level}
\item integrated variance of intensity surface
\item error rates in thresholding
\end{itemize}
\item minimize variance of coefficients for covariates
\end{itemize}

Ecology: mapping plants or animal nests using distance sampling
\begin{itemize}
\item minimize distance
\item minimize variance in parameters of detection function and/or point process
\item minimize one of these:
\begin{itemize}
\item maximum variance in intensity surface
\item integrated variance of intensity surface
\end{itemize}
\item minimize variance of coefficients for covariates
\end{itemize}

Heuristics of a good transect sampling plan
\begin{itemize}
\item Space-filling, criteria might be maximizing the path integral of nearest
neighbor distance along the transects
\item Should start with a sparse design with regular spacing, then refine with
infill
\begin{itemize}
\item Provides good spatial coverage even if aborted early
\item Imagine downloading a high-resolution intensity jpeg over 56k
\end{itemize}
\item Path should avoid sharp turns but is allowed to cross itself
\item One option is to generate two segments at a time, first a short-to-medium
length segment to get to the start of the next transect, then a  medium-to-long
segment for the transect
\item Could have new segment length be negatively correlated with the previous
segment length
\end{itemize}

add some citations: \cite{lark}, classical space-filling designs,
space-filling fractals


\section{Sampling Situations}

\begin{itemize}

\item SRS of parallel transects

\item systematic sample of parallel transects

\item fractal curves with random starting points

\item movement model
\begin{itemize}
\item generate sequentially, two waypoints at a time
\item generate a jump distance and a direction
\item distance negatively correlated with previous distance (should
approximately alternate between a short ``transition'' and a long ``transect'')
\item direction bimodal with modes near \(\pm \pi / 2\)
\item if using location/scale beta for direction, allow any direction when the
support 
\end{itemize}

\end{itemize}

set up to think about adaptive sampling (adding a transect at a time or
stopping early but don't actually do it here)


\bibliography{lgcp_sampling.bib}
\bibliographystyle{agsm}
\end{document}
